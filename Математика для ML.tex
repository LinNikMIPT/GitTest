\documentclass[12pt,a4paper]{article} 
\usepackage[utf8]{inputenc} 
\usepackage{amsmath}
\usepackage{amsfonts}
\usepackage{amssymb}
\usepackage{tipa}
\usepackage[left=2cm,right=2cm,top=2cm,bottom=2cm]{geometry}

% Русский язык 
\usepackage[T2A]{fontenc} % кодировка 
\usepackage[utf8]{inputenc} % кодировка исходного текста 
\usepackage[english,russian]{babel} % локализация и переносы 

% Математика 
\usepackage{amsmath,amsfonts,amssymb,amsthm,mathtools} 

\usepackage{wasysym} 
\date{Ноябрь 2018}
\author{Линдеман Никита}
\title{Математика для Machine Learning}
\DeclareMathOperator{\rg}{rg}
\DeclareMathOperator{\grad}{\textbf{grad}}


\begin{document}

\maketitle
\tableofcontents
\newpage

\section{Элементы математического анализа}
\subsection{Производная функции одной переменной}
\textbf{Определение 1.1.1} 
\textit{Производная} функции одной переменной $y = f(x)$ -- это предел:
\[ \frac{d}{dx}f(x) = \frac{dy}{dx} = f'(x) = \lim_{\Delta x \to 0} \frac{f(x + \Delta x) - f(x)}{\Delta x}. \]

\subparagraph{}
\underline{Пример}

Найдем по определению производную функции $y = x^2$ в точке $x_0 = 3$:
\[ y'(x) = \lim_{\Delta x \to 0} \frac{(x + \Delta x)^2 - x^2}{\Delta x} = \lim_{\Delta x \to 0} \frac{2x\Delta x + \Delta x^2}{\Delta x} = \lim_{\Delta x \to 0} (2x + \Delta x) = 2x\]
\[y'(x_0) = y'(3) = 2 \cdot 3 = 6\]

\subparagraph{}
Конечно, на практике никто не считает производные по определению, для этого есть готовые таблицы производных элементарных функций (которую полезно помнить наизусть) и правила дифференцирования.

\subparagraph{}
\textbf{Правила дифференцирования}:

1) Линейность производной: $(\alpha f(x) + \beta g(x))' = \alpha f'(x) + \beta g'(x)$

2) Дифференцирование произведения: $(f(x)g(x))' = f'(x)g(x) + f(x)g'(x)$


3) Дифференцирование частного: \[ \left(\frac{f(x)}{g(x)}\right)' = \frac{f'(x)g(x) - f(x)g'(x)}{g^2(x)} \]

4) Производная сложной функции: \[ (f(u))' = f'(u) \cdot u'\]

\subparagraph{}
\underline{Пример}

Найдем производную функции $ y(x) = \frac{\cos(x^2 + 1)}{\ln(1-2x) + 3} $:

\[ y'(x) = \frac{(\cos(x^2 + 1))' \cdot (\ln(1-2x) + 3) - (\cos(x^2 + 1)) \cdot (\ln(1-2x) + 3)'}{(\ln(1-2x) + 3)^2} \]

Отдельно найдем производные:
\[ (\cos(x^2 + 1))' = 2x \cdot (-\sin(x^2 + 1))\]
\[ (\ln(1-2x) + 3)' = -2 \cdot \frac{1}{1 - 2x}\]

Окончательно имеем:
\[ y'(x) = \frac{-2x\sin(x^2 + 1) \cdot (\ln(1-2x) + 3) + (\cos(x^2 + 1)) \cdot \frac{2}{1 - 2x}}{(\ln(1-2x) + 3)^2} \]

\subparagraph{}
Здесь стоит упомянуть про физический и геометрический смысл производной: ее можно интерпретировать как 
скорость изменения какой-то величины (например, маржинализм в экономике или скорость материального объекта в механике) или как тангенс угла наклона касательной к графику функции. 

Именно из интерпретации производной как тангенса угла наклона касательной следуют две важные прикладные теоремы, ради которых мы и ввели это понятие.

\subparagraph{}
\textbf{ТЕОРЕМА}
Функция примнимает свое локально максимальное или минимальное (экстремальное) значение только в тех точках, где ее производная равна нулю (обратное не всегда верно):
\[ y(x_0) \rightarrow min \hspace{5px} \Rightarrow \hspace{5px} y'(x_0) = 0,\]
\[ y(x_0) \rightarrow max \hspace{5px} \Rightarrow \hspace{5px} y'(x_0) = 0.\]

\subparagraph{}
\textbf{ТЕОРЕМА}
Функция $y = f(x)$ строго \textit{возрастает} на промежутке $(a, b)$, тогда и только тогда, когда в каждой точке этого промежутка производная $f'(x)$ строго положительна, или на метематическом языке:
\[ y = f(x) \nearrow, \hspace{5px} x \in (a, b) \Leftrightarrow\ \forall \hspace{5px} x \in (a, b) \hookrightarrow f'(x) > 0. \]
Функция $y = f(x)$ строго \textit{убывает} на промежутке $(a, b)$, тогда и только тогда, когда в каждой точке этого промежутка производная $f'(x)$ строго отрицательна, или на метематическом языке:
\[ y = f(x) \searrow, \hspace{5px} x \in (a, b) \Leftrightarrow\ \forall \hspace{5px} x \in (a, b) \hookrightarrow f'(x) < 0. \]


\subsection{Частные производные и градиент}
\textbf{Определение 1.2.1} 
Рассмотрим функцию $f(x_1, x_2, \ldots, x_n)$ от $n$ переменных $f: \mathbb{R}^n \longrightarrow \mathbb{R}$. \textit{Частная производная} $f(x_1, x_2, \ldots, x_n)$ по переменной $x_k$ -- это предел: \[ f'_{x_k} = \frac{\partial}{\partial x_k}f(x_1, \ldots, x_k, \ldots, x_n) = \lim_{\Delta x \to 0} \frac{f(x_1, \ldots, x_k + \Delta x, \ldots, x_n) - f(x_1, \ldots, x_k, \ldots, x_n)}{\Delta x}. \]

\subparagraph{}
Нахождение частных производных почти ничем не отличается от нахождения производной функции одной переменной (а чаще даже бывает и проще) -- просто надо найти производную функции $f(x_k)$ считая все
остальные переменные $ x_1, \ldots, x_{k - 1}, x_{k + 1}, \ldots, x_n$ константами.

\subparagraph{}
\underline{Пример}

Найдем дифференциал функции $f(x, y, z) = x \cos(2y - z) + xy$ по формуле
\[ df(x, y, z) = \frac{\partial f}{\partial x}dx + \frac{\partial f}{\partial y}dy + \frac{\partial f}{\partial z}dz. \]
Для этого найдем частные производные:
\[ \frac{\partial f}{\partial x} = f'_x =  \cos(2y - z) + y \]
\[ \frac{\partial f}{\partial y} = f'_y = 2x \cdot (- \sin(2y - z)) + x\]
\[ \frac{\partial f}{\partial z} = f'_z =  -x \cdot (- \sin(2y - z)) \]
Окончательно имеем:
\[ df(x, y, z) =  (\cos(2y - z) + y)dx + (-2x \cdot \sin(2y - z) + x)dy + (x \cdot \sin(2y - z))dz. \]

\textbf{Определение 1.2.2} 
\textit{Градиентом} функции $f(x_1, x_2, \ldots, x_n)$ от $n$ переменных называется $n$-мерный вектор, составленный из частных прозводных функции $f$:
\[\grad f(x_1, x_2, \ldots, x_n) = \begin{pmatrix}
f'_{x_1} \\
f'_{x_2} \\
\vdots \\
f'_{x_n}
\end{pmatrix}.
\]

\subparagraph{}
Смысл градиента таков: это вектор, своим направлением указывающий направление наибольшего возрастания функции $f(x_1, x_2, \ldots, x_n)$, значение которой меняется от одной точки пространства к другой, а по величине (модулю) равный скорости роста этой величины в этом направлении.

\subparagraph{}
\underline{Пример}

Найдем градиент функции $f(x_1, x_2, x_3, x_4) = x_2 \sin(x_1 x_3) + 2x_4$ в точке $(\frac{3 \pi}{2}, 0, 1, -2)$.

Частые производные:
\[ f'_{x_1} = x_2 x_3 \cos(x_1 x_3)\]
\[ f'_{x_2} = \sin(x_1 x_3)\]
\[ f'_{x_3} = x_2 x_1 \cos(x_1 x_3)\]
\[ f'_{x_4} = 2\]
Подставляя в частные производные соответствующие значения переменных, находим:
\[ \grad f = \begin{pmatrix}
0 \\
-1 \\
0 \\
2
\end{pmatrix}.
\]



\newpage
\section{Основы линейной алгебры}
\subsection{Векторы и матрицы}
\textbf{Определение 2.1.1} 
\textit{Вектор} -- это упорядоченный одномерный массив чисел. 

Мы будем обозначать векторы строчными латинскими буквами полужирным шрифтом и записывать их в виде вектор-столбца:
\[
\textbf{a} = \begin{pmatrix}
a_1 \\
a_2 \\
\vdots \\
a_m
\end{pmatrix}
\]

\subparagraph{}
\textbf{Определение 2.1.2} 
\textit{Матрица} -- это упорядоченный двумерный массив чисел. 

Множество матриц с $m$ строками и $n$ столбцами, то есть матрицы размеров $m \times n$ будем обозначать $\textbf{M}_{m \times n}$. Произвольную матрицу $\textbf{A} \in \textbf{M}_{m \times n}$
будем обозначать заглавными латинскими буквами и записывать в следующем виде: 
\[
A = \begin{pmatrix}
a_{11}& a_{12} &\ldots & a_{1n}\\
a_{21}& a_{22} &\ldots & a_{2n}\\
\vdots& \vdots &\ddots & \vdots\\
a_{m1}& a_{m2} &\ldots & a_{mn}
\end{pmatrix},
\]
где на пересечении $i$-той строки с $j$-тым стобцом будет находится элемент матрицы $a_{ij}$. То есть 
первый индекс элемента матрицы -- номер строки, второй -- номер столбца, в котором находится элемент.
\subparagraph{}
Заметим, что вектор -- частный случай матрицы: действительно, матрица размерами $m \times 1$ -- 
в точности и есть вектор-столбец. Значит, если мы научимся работать с матрицами, то срзу же получим и правила работы с векторами. Поэтому далее пока что не будем отдельно рассматривать векторы, а будем работать с матрицами произвольных размеров $m \times n$, считая, что возможны случаи $n = 1$. 

\subsection{Сложение и вычитание матриц, умножение матриц на число}
\textbf{Определение 2.2.1} 
Пусть даны две матрицы $A$, $B \in \textbf{M}_{m \times n}$: $A = (a_{ij})$, $B = (b_{ij})$. Матрица $C = (c_{ij})$ называется \textit{суммой матриц $A$ и $B$}, если $c_{ij} = a_{ij} + b_{ij}$ для любых
$i = \overline{1, m}$ и $j = \overline{1, n}$. Будем обозначать $A + B = C$.

\textbf{Определение 2.2.2} 
Пусть даны две матрицы $A$, $B \in \textbf{M}_{m \times n}$: $A = (a_{ij})$, $B = (b_{ij})$. Матрица $C = (c_{ij})$ называется \textit{разностью матриц $A$ и $B$}, если $c_{ij} = a_{ij} - b_{ij}$ для любых $i = \overline{1, m}$ и $j = \overline{1, n}$. Будем обозначать $A - B = C$.

\subparagraph{}
Очевидо, что при таком определении мы можем складывать только матрицы одного размера, и это достаточно важное замечание!

\subparagraph{}
\underline{Пример}
\[\begin{pmatrix}
1 && -4 && 0 \\
5 && 3 && -1 
\end{pmatrix} + \begin{pmatrix}
4 && 5 && -7 \\
0 && 2 && 1 
\end{pmatrix} = \begin{pmatrix}
5 && 1 && -7 \\
5 && 5 && 0 
\end{pmatrix}.
\]

\subparagraph{}
\textbf{Определение 2.2.3} 
Пусть дана матрица $A \in \textbf{M}_{m \times n}$: $A = (a_{ij})$ и число $\lambda$. Матрица $C = (c_{ij})$ называется \textit{произведением матрицы $A$ на число $\lambda$}, если $c_{ij} = \lambda a_{ij}$ для любых $i = \overline{1, m}$ и $j = \overline{1, n}$. Будем обозначать 
$\lambda A= C$.

\subparagraph{}
\underline{Пример}:
\[8 \cdot \begin{pmatrix}
1 && 7 \\
2 && -3 
\end{pmatrix} = \begin{pmatrix}
8 && 56  \\
16 && -24  
\end{pmatrix}.
\]

\subparagraph{}
Так как мы ввели сложение и вычитание матриц и умножение матриц на число покомпонентно, то все свойства чисел, связанные с этими операциями остаются верными и для матриц.

\subsection{Умножение матриц}
Пусть даны две матрицы $A \in \textbf{M}_{m \times n}$ и $B \in \textbf{M}_{n \times p}$: $A = (a_{ij})$, $B = (b_{ij})$. Матрица $C = (c_{ij}) \in \textbf{M}_{m \times p}$ называется \textit{произведением матриц $A$ и $B$}, если \[c_{ij} = \sum_{k=1}^n a_{ik}b_{kj}\] для любых $i = \overline{1, m}$ и $j = \overline{1, n}$. Будем обозначать $A \cdot B = C$.

\subparagraph{}
Из определения следует, что мы можем перемножать матрицы только если количество столбцов у первой
равно количеству строк у второй. Это важное замечание!

\subparagraph{}
\underline{Пример}
\[ A = \begin{pmatrix}
0 && -1 \\
4 && 5 \\
-8 && 6 \\
\end{pmatrix}, \hspace{3px}
B = \begin{pmatrix}
3 && 1\\
-2 && 0
\end{pmatrix}
\]

Найдем $C = A \cdot B$:

1. Найдем размер результирующей матрицы. Для этого запишем размеры перемножаемых матриц, в данном случае это $3 \times 2$ и $2 \times 2$. Убеждаемся, что количество столбцов у первой матрицы
равно количеству строк у второй, значит, мы можем перемножить матрицы. Размер результирующей матрицы
будет $3 \times 2$.

2. Последовательно найдем элементы результирующей матрицы $C = c_{ij}$. Для этого обозначим матрицы $A = (a_{ij})$, $B = (b_{ij})$:
\[ A = \begin{pmatrix}
a_{11}& a_{12} \\
a_{21}& a_{22} \\
a_{31}& a_{32} 
\end{pmatrix}, \hspace{3px}
B = \begin{pmatrix}
b_{11} && b_{12} \\
b_{21} && b_{22}
\end{pmatrix}
\]
Тогда по определению:
\[c_{11} = \sum_{k=1}^2 a_{1k}b_{k1} = a_{11}b_{11} + a_{12}b_{21} = 0 \cdot 3 + (-1) \cdot (-2) = 2 
\]

\[c_{21} = \sum_{k=1}^2 a_{2k}b_{k1} = a_{21}b_{11} + a_{22}b_{21} = 0 \cdot 1 + (-1) \cdot 0 = 0 
\]

\[c_{21} = \sum_{k=1}^2 a_{2k}b_{k1} = a_{21}b_{11} + a_{22}b_{21} = 4 \cdot 3 + 5 \cdot (-2) = 2 
\]

\[c_{22} = \sum_{k=1}^2 a_{2k}b_{k2} = a_{21}b_{12} + a_{22}b_{22} = 4 \cdot 1 + 5 \cdot 0 = 4 
\]

\[c_{31} = \sum_{k=1}^2 a_{3k}b_{k1} = a_{31}b_{11} + a_{32}b_{21} = (-8) \cdot 3 + 6 \cdot (-2) = -36 
\]

\[c_{32} = \sum_{k=1}^2 a_{3k}b_{k2} = a_{31}b_{12} + a_{32}b_{22} = (-8) \cdot 1 + 6 \cdot 0 = -8 
\]

3. Запишем ответ:
\[C = \begin{pmatrix}
c_{11} && c_{12} \\
c_{21} && c_{22} \\
c_{31} && c_{32}
\end{pmatrix} = \begin{pmatrix}
2 && 0 \\
2 && 4 \\
-36 && -8 
\end{pmatrix} = 2 \cdot \begin{pmatrix}
1 && 0 \\
1 && 2 \\
-18 && -4
\end{pmatrix} \]

\subparagraph{}
Таким образом, можно вывести мнемоническое правило для произведения матриц: чтобы получить элемент $c_{ij}$ результирующей матрицы, умножаем $i$-ую строку первой матрицы на $j$-ый столбец второй матрицы «почленно».

\subparagraph{}
\textbf{Свойства умножения матриц}:

1) Ассоциотивность: $A(BC) = (AB)C$

2) Некоммутативность в общем случае: $AB \neq BA$

3) Дистрибутивность: $A(B + C) = AB + AC$, $(A+B)C = AC + BC$

4) Ассоциативность и коммутативность относительно умножения на число: $(\lambda A)B = \lambda(AB) = A(\lambda B)$

\subsection{Транспонирование, скалярное произведение, длина вектора}
\textbf{Определение 2.4.1} 
Пусть дана матрица $A \in \textbf{M}_{m \times n}$: $A = (a_{ij})$. Матрица $C = (c_{ij}) \in \textbf{M}_{n \times m}$ называется \textit{транспонированной к $A$}, если $c_{ij} = a_{ji}$ для любых $i = \overline{1, m}$ и $j = \overline{1, n}$. Будем обозначать транспонированную к $A$ матрицу как $A^T$.

\subparagraph{}
\underline{Пример}
\[\begin{pmatrix}
1 && 2 && 3 \\
4 && 5 && 6
\end{pmatrix}^T = \begin{pmatrix}
1 && 4 \\
2 && 5 \\
3 && 6
\end{pmatrix}
\]

То есть при транспонировании матрицы $i$-ая строка становится $i$-ым столбцом, а $j$-ый столбец 
стоновится $j$-ой строкой.

\subparagraph{}
\textbf{Свойства транспонирования}:

1) $(A + B)^T = A^T + B^T$

2) $(A - B)^T = A^T - B^T$

3) $(\lambda A)^T = \lambda A^T$

4) $(AB)^T = B^TA^T$
 
\subparagraph{}
\textbf{Определение 2.4.2} 
\textit{Скалярное произведение векторов} 

\[
\textbf{a} = \begin{pmatrix}
a_1 \\
a_2 \\
\vdots \\
a_m
\end{pmatrix}, \hspace{3px} \textbf{b} = \begin{pmatrix}
b_1 \\
b_2 \\
\vdots \\
b_m
\end{pmatrix}
\]

это
\[\langle \textbf{a}, \textbf{b} \rangle = \textbf{a}^T \cdot \textbf{b} = \begin{pmatrix}
a_1 && a_2 && \ldots && a_n
\end{pmatrix} \cdot \begin{pmatrix}
b_1 \\
b_2 \\
\vdots \\
b_m
\end{pmatrix} = \sum_{k=1}^m a_{k} b_{k}\]

\subparagraph{}
\underline{Пример}

Найдем скалярное произведение вкекторов $\textbf{a}^T = (-2 \hspace{7px} 6 \hspace{7px} 0 \hspace{7px} 5)$ и $\textbf{b}^T = (4 \hspace{7px} 3 \hspace{7px} 8 \hspace{7px} -2)$:

По определению имеем:
\[\langle \textbf{a}, \textbf{b} \rangle = \sum_{k=1}^4 a_{k} b_{k} = (-2) \cdot 4 + 6 \cdot 3 + 0 \cdot 8 + 5 \cdot (-2) = 0\]
Мы получили, что векторы $\textbf{a}$ и $\textbf{b}$ ортогональны.

\subparagraph{}
\textbf{Определение 2.4.3} 
Два ненулевых вектора называются \textit{ортогональными}, если их скалярное произведение равно нулю.

\subparagraph{}
\textbf{Определение 2.4.4} 
\textit{Длина (модуль) вектора} $\textbf{a}^T = (a_1 \hspace{7px} a_2 \hspace{7px} \ldots \hspace{7px} a_m)$ -- квадратный корень из скалярного произведения этого вектора на себя, то есть:
\[a = |\textbf{a}| = \sqrt{\sum_{k=1}^m (a_k)^2}\]

\subparagraph{}
\underline{Пример}

Найдем длину вектора $\textbf{a}^T = (-2 \hspace{7px} 6 \hspace{7px} 0 \hspace{7px} 5)$:

По определению:
\[a = |\textbf{a}| = \sqrt{\sum_{k=1}^4 (a_k)^2} = \sqrt{(-2)^2 + 6^2 + 0^2 + 5^2} = \sqrt{65}\]

\subparagraph{}
\textbf{Определение 2.4.5} 
\textit{Скалярное произведение векторов} $\textbf{a}$ и $\textbf{b}$ -- это произведение модулей (длин) этих векторов на косинус угла $\alpha$ между ними:
\[ \langle \textbf{a}, \textbf{b} \rangle = ab \cdot \cos(\alpha)\]

\subparagraph{}
Мы привели два определения скалярного произведения векторов, что может вызывать некоторую путаницу -- ведь определения очень разные. На сомом деле эти определения эквивалентны (результат не будет зависить от способа вычисления), просто определение (2.4.5) более общее, а (2.4.2) более удобно при практическом  вычислении.

\subsection{Ранг и определитель матрицы}
\textbf{Определение 2.5.1} 
Система столбцов $A_1, A_2, \ldots, A_n$ из $\textbf{M}_{m \times 1}$ называется \textit{линейно зависимой}, если  некоторая их нетривиальная линейная комбинация равна нулевому столбцу, то есть 
если существуют числа $\lambda_1, \lambda_2, \ldots, \lambda_n$ такие, что хотя бы одно из этих чисел не нулевое (нетривиальность) и выполнено:
\[ \sum_{k = 1}^n A_k \lambda_k = O = \begin{pmatrix}
0 \\
0 \\
\vdots \\
0 \\
\end{pmatrix}  
\]

\subparagraph{}
\underline{Пример}

Докажем, что следующие столбцы линейно зависимы:
\[
\begin{pmatrix}
1 \\
2 \\
3 
\end{pmatrix}, \hspace{3px} \begin{pmatrix}
-4 \\
1 \\
0 
\end{pmatrix}, \hspace{3px} \begin{pmatrix}
-1 \\
7 \\
9 
\end{pmatrix}.
\]
Действительно, существует нетривиальная линейная комбинация этих столбцов, равная нулевому вектору (напомним, что мы отожествляем понятие матрицы размера $m \times 1$ с вектором):
\[
3 \cdot \begin{pmatrix}
1 \\
2 \\
3 
\end{pmatrix} + \begin{pmatrix}
-4 \\
1 \\
0 
\end{pmatrix} - \begin{pmatrix}
-1 \\
7 \\
9 
\end{pmatrix} = O.
\]

\subparagraph{}
\textbf{Определение 2.5.2} 
Целое неотрицительное число $r$ называется \textit{рангом матрицы} $A$ и обозначается $r = \rg A$, если из столбцов этой матрицы можно выбрать $r$ линейно независимых столбцов, но нельзя выбрать $r + 1$ линейно независимых стлбцов. 

\subparagraph{}
\textbf{Свойства ранга}:

1) $\rg A^T = \rg A$.

2) Если $A$ -- квадратная матрица и $|A| = 0$, то строки и столбцы матрицы линейно зависимы.

3) Линейная (не)зависимость столбцов матрицы эквивалентна линейной (не)зависимости строк.

4) Если все элементы строки или столбца матрицы умножить на отличное от нуля число, то ранг матрицы не изменится.

5) Ранг не изменится, если к элементам любой строки или столбца матрицы прибавить соответствующие элементы другой строки или столбца, умноженные на одно и то же число.

6) Ранг не изменится, если переставить два любых столбца или две строки матрицы.


\subparagraph{}
\textbf{Определение 2.5.3} 
\textit{Определитель (или детерминант)} матрицы размера $2 \times 2$ -- это
\[|A| = \det A = \begin{vmatrix}
a_{11} && a_{12} \\
a_{21} && a_{22} 
\end{vmatrix} = a_{11}a_{22} - a_{12}a_{21}.
\]

Детерминант матрицы размера $3 \times 3$ -- это 
\[|A| = \det A = \begin{vmatrix}
a_{11} && a_{12} && a_{13} \\
a_{21} && a_{22} && a_{23} \\
a_{31} && a_{32} && a_{33}
\end{vmatrix} = a_{11} \begin{vmatrix}
a_{22} && a_{23} \\
a_{32} && a_{33} 
\end{vmatrix} - a_{12} \begin{vmatrix}
a_{21} && a_{23} \\
a_{31} && a_{33} 
\end{vmatrix} + a_{13} \begin{vmatrix}
a_{21} && a_{22} \\
a_{31} && a_{32} 
\end{vmatrix} .
\]

Заметим, что детерминант определен только для квадратных матриц, то есть только для матриц размеров $n \times n$!

\subparagraph{}
\underline{Пример}

Найдем определитель:
\[ |A| = \det A = \begin{vmatrix}
0 && 1 && 3 \\
2 && 3 && 5 \\
3 && 5 && 7
\end{vmatrix} = 0 \cdot \begin{vmatrix}
3 && 5 \\
5 && 7
\end{vmatrix} - 1 \cdot \begin{vmatrix}
2 && 5 \\
3 && 7 
\end{vmatrix} + 3 \cdot \begin{vmatrix}
2 && 3 \\
3 && 5
\end{vmatrix} = 0 - (14 - 15) + 3 \cdot (10 - 9) = 4
\]

Мы получили, что наша матрица не вырождена.

\subparagraph{}
\textbf{Определение 2.5.4} 
Матрица называется \textit{вырожденой}, если ее определитель равен нулю.

\subparagraph{}
\textbf{Свойства детерминанта}:

1) $|A^T| = |A|$.

2) $|AB| = |A| \cdot |B|$.

3) Умножение всех элементов строки или столбца определителя на некоторое число равносильно умножению определителя на это число.

4) Если матрица содержит нулевую строку или столбец, то определитель этой матрицы равен нулю.

5) Если две строки или два столбца матрицы линейно зависимы, то определитель этой матрицы равен нулю.

6) При перестановке двух любых строк или столбцов определитель матрицы меняет знак.

7)  Определитель не изменится, если к элементам любой строки или столбца матрицы прибавить соответствующие элементы другой строки или столбца, умноженные на одно и то же число.

8) Определитель матрицы треугольного вида равен произведению элементов, стоящих на главной диагонали:

\[
|A| = \begin{vmatrix}
a_{11}& a_{12} &\ldots & a_{1n}\\
0 & a_{22} &\ldots & a_{2n}\\
\vdots& \vdots &\ddots & \vdots\\
0& 0 &\ldots & a_{nn}
\end{vmatrix} = \prod_{k = 1}^n a_{kk} = a_{11} \cdot a_{22} \cdot \ldots \cdot a_{nn} 
\].

\subparagraph{}
Заметим, что последнее свойство позволяет легко находить определитель матриц большого порядка, если перед этим привести матрицу к диоганальному виду (что всегда можно сделать, опираясь на предпоследнее свойство и используя алгоритм Гаусса).

\newpage
\section{Некоторые понятия теории вероятностей и математической статистики}
\subsection{Классическая вероятность}
\textbf{Определение 3.1.1} 
\textit{Дискретное вероятностное простнранство} -- это любой конечный набор элементов (\textit{элементарных исходов}) $\Omega = \{\omega_1, \omega_2, \ldots\omega_N\}$, каждому из которых сопоставлено некоторое положительное число $\omega_k \longmapsto P_k$ $(k = \overline{1, N})$, причем для этих чисел выполнено условие нормировки:
\[\sum_{k = 1}^N P_k = 1.\]

\textbf{Определение 3.1.2} 
Сопоставленное элементарному исходу $\omega_i$ число будем называть \textit{вероятностью} этого исхода и считать, что эта вероятность определена как предел отношения:
\[ P_i = \lim_{n \to \infty} \frac{n(\omega_i)}{n},\]
где $n(\omega_i)$ -- это количество испытаний, при котором исход $\omega_i$ наступил, а $n$ --  общее число испытаний.

\textbf{Определение 3.1.3}
\textit{Событие} -- это любое подмножество $A$ множества элементарных исходов $\Omega$: $A \subseteq \Omega$. 

\textbf{Определение 3.1.4}
\textit{Вероятность события} $A$ -- это сумма вероятностей тех элементарных исходов, которые составляют событие $A$:
\[P(A) = \sum_{i: \hspace{1px} \omega_i \in A} P_i. \]

\subparagraph{}
\underline{Пример}

Рассмотрим опыт с бросанием игральной кости. В нашей терминологии:

1) Вероятностное пространство -- это множество всевозможных элементарных исходов: $\Omega = \{ \omega_i $ $\textpipe$  $\omega_i = \{$выпала $i$-ая грань$\}$, $i = \overline{1, 6} \}$.

2) Так как при данном эксперименте ествественно предположить, что выпадение любой грани равновероятно, то с учетом условия нормировки естественно счтитать, что $i$-ая грань выпадет с вероятностью:
\[P_i = \frac{1}{6}.\]

3) Пример события -- выпала четная грань. Действительно, выпадение четной грани есть подмножество     $A = \{\omega_2, \omega_4, \omega_6 \}$ всех элементарных исходов $\Omega$. Тогда вероятность такого события -- сумма:
\[P(A) = \sum_{i: \hspace{1px} \omega_i \in A} P_i = \sum_{i = 1}^3 P_i = \frac{1}{6} + \frac{1}{6} + \frac{1}{6} = \frac{1}{3}. \]

\subsection{Случайные величины}
\textbf{Определение 3.2.1}\\*
Случайная величина $\xi$ - это любая функция из $\Omega$ в $R$, т.е.
$\xi: \Omega \xrightarrow{} R$
\\*
\\*
Считаем, что случайная величина может принимать значения $\xi \sim (x_1, x_2, ..., x_m)$, где $m \leqslant N(\Omega)$ и $x_i < x_{i+1}$. Тогда вероятность того, что $\xi$ принимает некоторое значение $x_k$ равна сумме вероятностей событий $\omega_i$ для которых $\xi(\omega_i) = x_k$ или 
\\*\[P_{\xi}(k) = \sum_{i: \xi(\omega_i) = x_k } P_i\]
\\*где $P_{\xi}(k)$ = $P(\xi = x_k)$ по обозначению.
\\*
Таким образом, случайную величину удобно интерпритировать как результат какого-либо случайного события, т.е. $\omega_i \xrightarrow{} \xi(\omega_i)$/. 
\\* \\*
\underline{Пример}\\*
(пример с костью, где $\xi =$ число, выпавшее на кубике)\\*
\textbf{Определение 3.2.2}\\*
Математическим ожиданием (или мат.ожиданием) случайной величины называют величину
\\*\[E(\xi)=\sum^{N(\Omega)}_{i=1} P_i\xi(\omega_i)\]
\\* По сути мат.ожидание - это некое среднее значение случайной величины при достаточно большом количестве измерений (Закон Больших Чисел).\\*\\*
\textbf{Свойства мат.ожидания}\\*
1)$E(c\xi) = cE(\xi)$\\*
2)$E(\xi + \eta) = E(\xi) + E(\eta)$\\*
3)$\xi \geqslant 0 \Rightarrow E(\xi) \geqslant 0$\\*
4)$\xi \geqslant \eta \Rightarrow E(\xi) \geqslant E(\eta)$ \\*
\\*
\textbf{Определение 3.2.3}
\\*Дисперсийей случайной величины называется величина $D(\xi) = E((\xi - E(\xi))^2)= E(\xi^2) - (E(\xi))^2$\\*
Смысл имеет квадратный корень дисперсии $\sigma(\xi) = \sqrt{D(\xi)}$ - среднеквадратичное отклонение случайной величины. $\sigma(\xi)$ характеризует отклонение случайно величины от ее математического ожидания. 
\\*
\textbf{Свойства дисперсии}\\*
1)$D(c\xi) = c^2D(\xi)$\\*
2)$D(\xi) \geqslant 0$,  $D(\xi) = 0 \Leftrightarrow \xi(\omega_i) = E(\xi)$  $\forall i$\\*
3)$D(\xi + \eta) \geqslant D(\xi) + D(\eta) $

\end{document}


\end{document}
